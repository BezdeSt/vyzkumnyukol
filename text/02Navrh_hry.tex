\section{Návrh strategické hry}

\begin{itemize}
    \item Představení cíle kapitoly.
    \item Vysvětlení, proč je důležité systematicky navrhnout jednotlivé prvky hry.
    \item Úvodní představy
    \begin{itemize}
        \item Herní koncept (typ hry, pravidla, cíle hráče).
        \item Popis mapy: Použití dlaždic a jejich vliv na herní mechaniky.
    \end{itemize}
    \item Odkaz na \textit{The Art of Game Design} a použitý rámec návrhu.
    \item Přehled toho, co bude v kapitole popsáno.
\end{itemize}

\subsection{Prostor}
\begin{itemize}
    \item Struktura herního světa a jeho reprezentace.
    \item Herní mapa:
    \begin{itemize}
        \item Dlaždicová struktura a její výhody.
        \item Typy terénu a jejich vlastnosti.
    \end{itemize}
    \item Pohyb pouze ve čtyřech směrech
\end{itemize}

\subsection{Objects, Attributes, and States (Objekty, atributy a stavy)}
\begin{itemize}
    \item Herní entity a jejich stavy:
    \begin{itemize}
        \item Dlaždice (Typy terénů, Stavy dlaždic)
        \begin{itemize}
            \item Typy terénu a jejich vlastnosti
            \item Stav dlaždice -- obsazená, zastavěná
        \end{itemize}
        \item Jednotky
        \begin{itemize}
            \item Typ jednotky -- vlastnosti (pohyb, útok, obrana, speciální schopnosti)
            \item Stavy jednotek -- zraněná, pracuje, ...
        \end{itemize}
        \item Budovy (produkce, obranné stavby, speciální budovy).
        \begin{itemize}
            \item Vlastnosti budov -- cena, produkce
            \item Speciální akce budov
        \end{itemize}
    \end{itemize}
\end{itemize}

\subsection{Actions (Akce hráče)}
\begin{itemize}
    \item Možné akce hráče během tahu:
    \begin{itemize}
        \item Pohyb jednotek. (Použití speciálních schopností jednotek)
        \item Útok a bojový systém. 
        \item Stavba budov a infrastruktury. (Správa zdrojů a ekonomika)
    \end{itemize}
\end{itemize}

\subsection{Rules (Pravidla hry)}
\begin{itemize}
    \item Základní pravidla:
    \begin{itemize}
        \item Podmínky vítězství (zničení nepřítele).
        \item Struktura tahu (pořadí akcí).
        \item Omezující pravidla (pohybové limity, ztráta jednotek, zničení budov).
    \end{itemize}
    \item Vyváženost a spravedlnost pravidel.
\end{itemize}

\subsection{Skill and Chance (Dovednost a náhoda)}
\begin{itemize}
    \item Jaký vliv má dovednost hráče na výsledek hry:
    \begin{itemize}
        \item Strategické plánování.
        \item Správné rozhodování a reakce na situace.
    \end{itemize}
    \item Jaký vliv má náhoda:
    \begin{itemize}
        \item Procedurálně generovaná mapa jako faktor variabilních podmínek.
    \end{itemize}
    \item Vyvážení mezi dovedností a náhodou.
\end{itemize}

\subsection{Shrnutí}
\begin{itemize}
    \item Shrnutí hlavních bodů návrhu.
    \item Vztah návrhu k procedurálnímu generování map.
\end{itemize}
