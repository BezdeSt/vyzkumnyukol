\section{Návrh strategické hry}

\begin{itemize}
    \item Představení cíle kapitoly.
    \item Vysvětlení, proč je důležité systematicky navrhnout jednotlivé prvky hry.
    \item Úvodní představy
    \begin{itemize}
        \item Herní koncept (typ hry, pravidla, cíle hráče).
        \item Popis mapy: Použití dlaždic a jejich vliv na herní mechaniky.
    \end{itemize}
    \item Odkaz na \textit{The Art of Game Design} a použitý rámec návrhu.
    \item Přehled toho, co bude v kapitole popsáno.
\end{itemize}

Zpracováno na základě: \cite{} \cite{} \cite{}.

V této kapitole rozeberu postup návrhu jednoduché strategické hry hrané na náhodně vygenerované mapě.

Návrh strategické hry vyžaduje systematický přístup k definování jejích klíčových prvků. Aby byla hra hratelná, vyvážená a pokud možno i zábavná, je zapotřebí systematicky rozebrat a promyslet návrh všech částí hry. 

Základní představa, se kterou jsem začínal před samotným systematickým návrhem, byla taková, že hra bude 2D tahová strategie s prvky správy surovin. Hráči budou ovládat a vylepšovat své jednotky, spravovat zdroje a budovat infrastrukturu, přičemž se budou snažit dosáhnout vítězství nad soupeřem. Herní svět je reprezentován dlaždicovou mapou, typ terénu na dlaždici ovlivňuje pohyb jednotek, možnosti výstavby a dostupnost zdrojů.

Pro návrh hry využívám metodologii z knihy \textit{The Art of Game Design: A Book of Lenses} od Jesseho Schella, která strukturuje herní design do několika klíčových kategorií:

\begin{itemize}
    \item Prostor -- Jak je herní svět uspořádán a jak se v něm hráči pohybují.
    \item Objekty, atributy a stavy -- Jaké herní entity existují, jaké mají vlastnosti a jaké stavy mohou při hře nastat.
    \item Akce hráče -- Jaké interakce může hráč provádět a jak ovlivňují hru.
    \item Pravidla hry -- Jaké jsou omezení a podmínky vítězství.
    \item Dovednost a náhoda -- Jaký je poměr mezi strategickým rozhodováním a prvky náhody.
\end{itemize}

Tento rámec poskytuje ucelený pohled na návrh hry a pomáhá zajistit, aby všechny prvky dohromady tvořily soudržný a dobře vyvážený celek.

Ve zbytku kapitoly rozeberu podrobněji právě tyto herní prvky a konkretizuji herní návrh.

\subsection{Prostor}
\begin{itemize}
    \item Struktura herního světa a jeho reprezentace.
    \item Herní mapa:
    \begin{itemize}
        \item Dlaždicová struktura a její výhody.
        \item Typy terénu a jejich vlastnosti.
    \end{itemize}
    \item Pohyb pouze ve čtyřech směrech
\end{itemize}

Každá hra obsahuje nějaký herní prostor, ve kterém se celá hra odehrává. Ten obecně vymezuje herní lokace a určuje, jak jsou mezi sebou propojeny. Z pohledu herních mechanik považujeme prostor za matematickou konstrukci, tedy je potřeba odfiltrovat veškeré vizuální prvky a zaměřit se pouze na abstraktní uspořádání prostoru.

Kniha \textit{The Art of Game Design: A Book of Lenses} rozděluje herní prostory podle několika parametrů:

\begin{itemize}
    \item Diskrétnost vs. kontinuita -- Prostor může být buď diskrétní, nebo kontinuální. Diskrétní prostor je tvořen pevně stanovenými, oddělenými místy, kontinuální prostor umožňuje pohyb v plynulém, nekonečném rozsahu.
    \item Počet dimenzí -- Každý herní prostor má určitý počet dimenzí, které definují jeho rozsah a strukturu. Prostor může být jednorozměrný, dvourozměrný nebo dokonce třírozměrný, jak ukazuje například herní stůl na kulečník.
    \item Ohraničenost a propojení oblastí -- Prostor může být uzavřený, s pevně definovanými hranicemi, nebo otevřený, umožňující pohyb hráče nebo herních prvků mimo hranice. Rovněž je důležité zvážit, zda jsou jednotlivé části prostoru propojené, nebo zda jsou oddělené a nezávislé.
\end{itemize}

Příklad hry hrané na diskrétním dvoudimenzionálním poli je "piškvorky" (v rámci příkladu předpokládám herní plochu $3\times3$), kde herní plocha je rozdělena na devět oddělených políček. Herní deska je zobrazena jako souvislý prostor, ale z hlediska herních mechanik bereme v potaz pouze těchto devět specifických míst, která můžeme znázornit jako uzly v síti. Tedy dokud je jednoznačně rozeznatelné, do kterého políčka zapadá, jsou si mechanicky ekvivalentní, jak je naznačeno na obrázku \ref{sudoku}.

\begin{figure}
  \centering      % vycentrovat
  \includegraphics[scale=0.3]{obr/sudoku.png} % soubor + měřítko (scale)
  \caption{Ekvivalence různých značení sudoku.} % popis obrázku
  \label{sudoku} % definice odkazu na obrázek (pro \ref{})
\end{figure}

Hru monopoly pak můžeme definovat jako příklad jednodimenzionální hry. I když je herní deska vizuálně uspořádána do tvaru čtverce, když odstraníme grafické prvky, můžeme vidět, že hra umožňuje pohyb po jediné řadě políček, propojených v cyklické smyčce. V tomto případě se každé políčko na desce chová jako bod nulové dimenze, ačkoliv vizuálně vypadají některé čtverce odlišně, jejich funkce se neliší.

Herní prostor může také zahrnovat „prostory v prostorech“, což se často vyskytuje v počítačových hrách. Například může existovat venkovní prostor (kontinuální, dvourozměrný), ale hráč může narazit na ikony, které představují města nebo jeskyně. Tyto ikony přecházejí do zcela oddělených prostorů, které s venkovním prostorem nejsou přímo propojené, což je příkladem prostorového uspořádání, které je více založeno na mentálních modelech hráčů než na geografické realitě.

\subsubsection{Návrh herního prostoru}

Jak jsem již zmínil, navrhovaná hra bude 2D tahová strategie, tedy herní prostor bude dvoudimenzionální a diskrétní, složený ze čtvercových políček propojených po horizontální a vertikální ploše. To bude mít zásadní vliv na pohyb jednotek po mapě a tedy i veškerá strategická rozhodnutí hráčů.

Diskrétnost herního prostoru usnadňuje měření vzdáleností, po které se jednotky mohou po desce pohybovat, a také omezuje rozložení budov na jednu budovu na políčko. Stejně tak pro jednotky, což umožňuje strategické tahy, jako blokování pohybu jednotek nepřítele.

Ve hře nebudou žádné podprostory ani oblasti, které by hráči mohli navštívit jako separátní oblasti. Celý herní prostor tvoří jednotnou plochu bez vnitřních "meziprostorů" nebo propojení do nových sekcí. Všechny interakce probíhají na jednom herním poli, přičemž všechny herní mechaniky se soustředí na strategické využití této plochy.

Z hlediska návrhu a pozdější implementace tedy herní prostor uvažuji jako dvourozměrné diskrétní pole, kde každé "políčko" představuje bod s nulovou dimenzí. Tento pohled by měl zjednodušit návrh pravidel a interakcí mezi hráči a prostředím, což umožňuje efektivnější rozvoj herních taktik.

\subsubsection{Shrnutí prostoru}

Herní pole tedy pro účely návrhu uvažuji jako abstraktní dvoudimenzionální diskrétní prostor složený z políček propojených na ose $x$ a $y$. Hráči tedy budou moci pohybovat jednotkami pouze ve vertikálním či horizontálním směru, ne diagonálně. Prostor bude jasně vymezen hranicemi (konec vygenerované mapy) a všechny interakce mezi hráči budou probíhat na herní ploše. 

\subsection{Objects, Attributes, and States (Objekty, atributy a stavy)}
\begin{itemize}
    \item Herní entity a jejich stavy:
    \begin{itemize}
        \item Dlaždice (Typy terénů, Stavy dlaždic)
        \begin{itemize}
            \item Typy terénu a jejich vlastnosti
            \item Stav dlaždice -- obsazená, zastavěná
        \end{itemize}
        \item Jednotky
        \begin{itemize}
            \item Typ jednotky -- vlastnosti (pohyb, útok, obrana, speciální schopnosti)
            \item Stavy jednotek -- zraněná, pracuje, ...
        \end{itemize}
        \item Budovy (produkce, obranné stavby, speciální budovy).
        \begin{itemize}
            \item Vlastnosti budov -- cena, produkce
            \item Speciální akce budov
        \end{itemize}
    \end{itemize}
\end{itemize}

V herním prostoru se pak nacházejí objekty se kterými hráči v průběhu celé hry manipulují nebo s nimi interagují pomocí jiných objektů. Může se jednat například o herní figurky, postavy, karty, nebo samotné herní prostředí. Objekty lze chápat jako „podstatná jména“ herních mechanik. 

Každý objekt má nějaké atributy, které popisují jeho vlastnosti. Atributy lze chápat jako "přídavná jména". Například auta v závodní hře mohou mít atributy jako \textit{maximální rychlost} a \textit{aktuální rychlost}. 

Každý herní objekt má \textbf{stav}, který určuje jeho vlastnosti a chování ve hře. Stav objektu se skládá z jeho atributů, což jsou jednotlivé charakteristiky objektu. Například figurka v šachu má atribut „mód pohybu“, který může nabývat stavů jako „volně se pohybující“, „v šachu“ a „matovaná“. V Monopoly má každý pozemek atribut „počet domů“, jehož stav se může měnit mezi 0 až 4 domy nebo hotelem.

Atributy mohou být statické nebo dynamické:

\begin{itemize} \item Statické atributy – Nemění se v průběhu hry. Například barva figurky v dámě nebo maximální rychlost auta.
\item Dynamické atributy – Mohou se měnit v závislosti na akcích hráčů či mechanikách hry. Například aktuální rychlost auta se mění podle řízení hráče, životy jednotky klesají při útoku. \end{itemize}

Každý herní objekt tak může mít kombinaci statických a dynamických atributů. Například v šachu má figurka statický atribut „barva“, ale dynamický atribut „pozice na šachovnici“, který se mění během hry.

Je důležité si uvědomit, že objekty ve hře často interagují mezi sebou. Například, jednotka může útočit na jinou jednotku, budova může produkovat suroviny, nebo terénní prvek může ovlivňovat pohyb jednotek. Tyto interakce by měly být navrženy tak, aby dávaly smysl a byly pro hráče srozumitelné.

Zásadní změny ve stavu herních objektů, ať už se jedná o změnu atributů nebo interakce s jinými objekty, je potřeba hráči nějakým způsobem indikovat. Vizuální reprezentace stavů objektů by měla být hráčům srozumitelná a intuitivní.

Stav objektu není nutně viditelný celý – některé atributy mohou být skryté nebo viditelné jen pro určité hráče.

Důležitou součástí herního návrhu tedy je rozhodnout, kdo má přístup k jakým informacím. Pro jednoduchost rozlišíme informace na \textit{veřejné} \textit{částečně skryté} nebo \textit{zcela skryté}. 

\begin{itemize}
    \item Veřejné informace -- Všechny atributy a jejich stavy jsou viditelné pro všechny hráče. Například v šachu oba hráči vidí všechna pole a figurky na hrací desce, takže jediným tajemstvím je přemýšlení soupeře.
    \item Částečně skryté informace -- Někteří hráči znají určitou informaci, ale jiní ne. Například ve hře poker někteří hráči viděli kartu, zatímco jiní ne.
    \item Zcela skryté informace -- Existují atributy, které zná pouze samotná hra. Například v počítačových hrách mohou být některé části světa před hráčem skryté, dokud je neodhalí.
\end{itemize}

Rozhodnutí o tom, kdo má přístup k jakým informacím, zásadně ovlivňuje herní strategii a atmosféru. Hry jako poker jsou postavené na utajení a odhadu soupeřových karet, zatímco v šachu mají hráči k dispozici informace o stavu celé herní plochy, a jedinou neznámou je strategie protivníka. Změna dostupnosti informací může radikálně proměnit hratelnost, například když se dříve skrytá informace náhle odhalí a změní dynamiku hry.

\subsubsection{Návrh herních objektů}

V této části se podle probraných principů pokusím nastínit jednotlivé objekty a definovat atributy těchto objektů.

Ve hře se nachází několik typů objektů, se kterými hráči mohou manipulovat nebo s nimi interagovat. Základním objektem ve hře jsou \textbf{políčka} ze kterých je složené herní pole a určují podmínky pro pohyb jednotek a stavbu budov. \textbf{Budovy}, produkují suroviny, případně poskytují další služby a plní jiné strategické funkce. Jako poslední jsou \textbf{jednotky} což jsou pohyblivé objekty ovládané hráčem, které hráč využívá k různým činnostem jako je boj, nebo těžba surovin. Každý z těchto objektů má své specifické atributy a stavy, které určují jejich vlastnosti a možnosti ve hře.

Co se informací týče nakonec jsem se rozhodl, že hra nebude mít skryté informace, tedy hráči od začátku uvidí celý herní prostor a pohyby protivníka.


\paragraph{Políčka}
Políčka představují základní stavební jednotku mapy, na níž se odehrává hra. Každé políčko má několik atributů, popisujících jeho vlastnosti a interakce s ostatními objekty. Tyto atributy jsou:
\begin{itemize}
    \item Statické:
    \begin{itemize}
        \item \textbf{Pozice na mapě} -- Souřadnice určující umístění políčka v herním prostoru.
        \item \textbf{Název} -- Název terénu.
        \item \textbf{Zpomalení} -- Určité typy terénů mohou snižovat pohybovou rychlost jednotek. 
        \item \textbf{Bonusová obrana} -- Některé terény mohou (např. Hory) mohou zvyšovat obranu jednotek na daném políčku.
        \item \textbf{Získatelné suroviny} -- Typ surovin které je možné na políčku získat.
    \end{itemize}
    \item Dynamické:
    \begin{itemize}
        \item \textbf{Obsazenost jednotka} -- Na políčku se může nacházet pouze jedna jednotka, tento atribut tedy bude bránit vstupu jiné jednotky na políčko.
    \item \textbf{Obsazenost budova} -- Na políčku může stát pouze jedna budova.
    \end{itemize}
%    \item \textbf{Možné zdroje} – některá políčka obsahují suroviny jako dřevo, kámen nebo jídlo. 
%    \item \textbf{Zpomalení} – některé terény zpomalují pohyb jednotek.
\end{itemize}

Terén není jen grafický prvek, ale významně ovlivňuje hru. Správná volba umístění jednotek může znamenat rozdíl mezi vítězstvím a porážkou – například jednotka stojící na horách má lepší obranu, zatímco husté lesy mohou zpomalit postup nepřátel. Navíc různé druhy terénu určují, jaké budovy lze postavit a jaké suroviny lze těžit.



\paragraph{Budovy}
Budovy jsou struktury, které hráči staví na mapě za účelem generování zdrojů nebo poskytování jiných výhod. Každá budova má následující atributy:
\begin{itemize}
    \item Statické:
    \begin{itemize}
        \item \textbf{Pozice na mapě} -- Pozice budovy na herní mapě.
        \item \textbf{Název} -- Označení budovy.
        \item \textbf{Vlastník} -- Určuje hráče, kterému budova patří. Změna vlastníků během hry nebude možná, poze zničení nepřátelských budov.
        \item \textbf{Typ terénu} -- Omezení, na kterých typech políček lze budovu postavit.
        \item \textbf{Produkce za kolo} – množství surovin, které budova generuje za kolo.
        \item \textbf{Životy max} -- Maximální počet životů budovy.
        \item \textbf{Obrana} -- O se zredukuje poškození způsobené útokem.
        \item \textbf{Cena} -- Množství surovin potřebných pro stavbu budovy.
        \item \textbf{Bonusová obrana} -- Budova může zvyšovat obranu jednotek nacházejících se na stejném políčku.
        \item \textbf{Speciální funkce} -- Budova může umožňovat provádění speciálních akcí jako generování nebo vylepšování jednotek.
    \end{itemize}
    \item Dynamické:
    \begin{itemize}
        \item \textbf{Životy} -- Aktuální počet životů budovy, pokud klesne na nulu budova je zničena.
    \end{itemize}
\end{itemize}
Budovy kromě generování surovin poskytují hráči jiné strategické možnosti jako vylepšování jednotek, zvyšování obrany vlastních jednotek nebo blokování postupu nepřítele.

\paragraph{Jednotky}
Jednotky jsou pohyblivé objekty na mapě, které hráči ovládají a skrze které primární interagují se s herními mechanismy. Každá jednotka má své atributy, které určují její vlastnosti a schopnosti:
\begin{itemize}
    \item Statické:
    \begin{itemize}
        \item \textbf{Název} -- Název typu jednotky.
        \item \textbf{Pozice na mapě} -- Kde na herním poli se jednotka nachází.
        \item \textbf{Vlastník} – hráč, kterému jednotka patří.
        \item \textbf{Cena} -- Množství surovin potřebné pro vytvoření jednotky.
        \item \textbf{Cena za kolo} -- Náklady na udržování jednotky. Množství surovin které jednotka spotřebuje každé kolo.
        \item \textbf{Životy max} -- Maximální počet životů jednotky.
         \item \textbf{Základní obrana} -- Základní obrana jednotky. Redukuje poškození způsobené nepřátelským útokem.
        \item \textbf{Útok} -- Síla útoku. Určuje o kolik se sníží životy nepřátelské jednotky při útoku. Poškození je redukované obranou. 
        \item \textbf{Dosah} -- Vzdálenost, na kterou může jednotka útočit.
        \item \textbf{Základní rychlost} – Maximální počet políček, která může jednotka urazit za kolo.
    \end{itemize}
    \item Dynamické:
    \begin{itemize}
        \item \textbf{Životy} -- Aktuální počet životů jednotky, pokud klesne na nulu jednotka zmizí.
        \item \textbf{Obrana} -- Funkční obrana jednotky, po modifikaci prostředím (Typem terénu nebo budovou.).
        \item \textbf{Rychlost} -- Skutečný počet políček přes který se jednotka může v daném tahu pohybovat, po modifikaci terénem.
        \item \textbf{Zaměstnaná} -- Indikuje, zda jednotka vykonala akci v tomto tahu.
        \item \textbf{V pohybu} -- Indikuje zda se jednotka v tahu pohybovala.
        \item \textbf{Zkušenosti} -- Jednotka může být vylepšena v konkrétní budově po dosažení určitého počtu zkušeností, které získává prováděním akcí odpovídajících jejímu typu (válečníci získávají zkušenosti bojem, pracovníci těžbou surovin nebo opravami budov).
    \end{itemize}
\end{itemize}

Jednotky představují hlavní způsob, jak hráč ovlivňuje dění na mapě. Každá jednotka má specifickou roli – některé slouží k boji, jiné k těžbě surovin nebo stavbě budov. Postupem času mohou získávat zkušenosti a vylepšovat své schopnosti, což přidává další vrstvu strategického rozhodování. Kromě toho jednotky také každé kolo spotřebovávají množství surovin, tedy hráč musí zvážit zda si může dovolit postavit velkou armádu slabých jednotek.



\subsection{Actions (Akce hráče)}
\begin{itemize}
    \item Možné akce hráče během tahu:
    \begin{itemize}
        \item Pohyb jednotek. (Použití speciálních schopností jednotek)
        \item Útok a bojový systém. 
        \item Stavba budov a infrastruktury. (Správa zdrojů a ekonomika)
    \end{itemize}
\end{itemize}

\subsection{Rules (Pravidla hry)}
\begin{itemize}
    \item Základní pravidla:
    \begin{itemize}
        \item Podmínky vítězství (zničení nepřítele).
        \item Struktura tahu (pořadí akcí).
        \item Omezující pravidla (pohybové limity, ztráta jednotek, zničení budov).
    \end{itemize}
    \item Vyváženost a spravedlnost pravidel.
\end{itemize}

\subsection{Skill and Chance (Dovednost a náhoda)}
\begin{itemize}
    \item Jaký vliv má dovednost hráče na výsledek hry:
    \begin{itemize}
        \item Strategické plánování.
        \item Správné rozhodování a reakce na situace.
    \end{itemize}
    \item Jaký vliv má náhoda:
    \begin{itemize}
        \item Procedurálně generovaná mapa jako faktor variabilních podmínek.
    \end{itemize}
    \item Vyvážení mezi dovedností a náhodou.
\end{itemize}

\subsection{Shrnutí}
\begin{itemize}
    \item Shrnutí hlavních bodů návrhu.
    \item Vztah návrhu k procedurálnímu generování map.
\end{itemize}
