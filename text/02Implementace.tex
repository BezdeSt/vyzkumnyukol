\definecolor{darkgreen}{rgb}{0.0, 0.5, 0.0}   % Barva pro stringy
\definecolor{darkgray}{rgb}{0.3, 0.3, 0.3}    % Barva čísel řádků
\definecolor{codeblue}{rgb}{0.0, 0.0, 0.6}    % Barva pro klíčová slova
\definecolor{codegray}{rgb}{0.5, 0.5, 0.5}    % Neutrální šedá
\definecolor{commentgreen}{rgb}{0.0, 0.6, 0.0} % Barva pro komentáře

\lstdefinestyle{mystyle}{
    backgroundcolor=\color{white}, % pozadí kódu
    commentstyle=\color{commentgreen}\itshape,  % Celý komentář zeleně + kurzíva
    keywordstyle=\color{codeblue}\bfseries,     % Klíčová slova modře a tučně
    numberstyle=\tiny\color{darkgray},          % Čísla řádků šedě
    stringstyle=\color{darkgreen},              % Řetězce tmavě zeleně
    basicstyle=\ttfamily\footnotesize,          % Základní styl
    breakatwhitespace=false,
    breaklines=true,
    captionpos=b,
    keepspaces=true,
    numbers=left,
    numbersep=5pt,
    showspaces=false,
    showstringspaces=false,
    showtabs=false,
    tabsize=2,
    frame=single,
    extendedchars=true,
    morecomment=[l]{\#},
    literate=
      {á}{{\'a}}1 {č}{{\v{c}}}1 {ď}{{\v{d}}}1 {é}{{\'e}}1 {ě}{{\v{e}}}1 {í}{{\'i}}1
      {ň}{{\v{n}}}1 {ó}{{\'o}}1 {ř}{{\v{r}}}1 {š}{{\v{s}}}1 {ť}{{\v{t}}}1 {ú}{{\'u}}1
      {ů}{{\r{u}}}1 {ý}{{\'y}}1 {ž}{{\v{z}}}1
      {Á}{{\'A}}1 {Č}{{\v{C}}}1 {Ď}{{\v{D}}}1 {É}{{\'E}}1 {Ě}{{\v{E}}}1 {Í}{{\'I}}1
      {Ň}{{\v{N}}}1 {Ó}{{\'O}}1 {Ř}{{\v{R}}}1 {Š}{{\v{S}}}1 {Ť}{{\v{T}}}1 {Ú}{{\'U}}1
      {Ů}{{\r{U}}}1 {Ý}{{\'Y}}1 {Ž}{{\v{Z}}}1
}
\lstset{style=mystyle}

% Změna textu "Listing" na "Ukázka"
\renewcommand\lstlistingname{Ukázka}


\section{Generování herního pole}

Pro generování herního pole jsem zvolil metodu výškových map. Výšková mapa je dvourozměrná matice reálných čísel, kde každé číslo představuje výšku v daném bodě. Výškovou mapu následně převedu na konkrétní terénní typy, jako jsou voda, pláně, lesy a hory.

V této kapitole popíšu implementaci generování výškových map třemi způsoby:

\begin{itemize}
    \item Náhodná výšková mapa -- nejjednodušší metoda, která však vytváří nerealistický terén.
    \item Interpolace hrubé mřížky -- metoda, která vyhlazuje terén pomocí interpolace mezi body náhodné mřížky.
    \item Gradientní šum -- generování přirozeně vypadajícího terénu pomocí Perlinova šumu.
\end{itemize}

\section{Implementované metody}

Výškové mapy jsem, jak už bylo zmíněno, implementoval třemi různými způsoby.

\subsection{Náhodná výšková mapa}

Tuto metodu jsem implementoval především proto, že je velmi jednoduchá a může sloužit jako základní "benchmark" pro srovnání s ostatními metodami. Výsledky této metody nejsou příliš dobré, protože nevytváří realistické terénní struktury. Výsledné struktury nejsou realistické ani praktické pro hraní.

Celá implementace spočívá v generování náhodných hodnot a jejich uložení do dvourozměrného pole, které pak funkce vrací \ref{kod_nahodne_pole}.


\begin{lstlisting}[language=Python, caption=Kód generující pole pro náhodnou mapu, label=kod_nahodne_pole]
def nahodne_pole(rows, cols, min_value=0, max_value=1):
    return np.random.uniform(low=min_value, high=max_value, size=(rows, cols))

\end{lstlisting}


\subsection{Interpolovaná výšková mapa}

Lepšího výsledku lze dosáhnout interpolací náhodné mřížky. Tato metoda spočívá v tom, že vygeneruji menší náhodnou mřížku, její hodnoty rozmístím pravidelně do větší mřížky a prázdné hodnoty pak získám interpolací hodnot z menší mřížky. Pro samotnou interpolaci využívám v kódu \ref{kod_interpolovane_pole} knihovnu \textit{scipy}. Funkce nakonec opět vrací matici čísel mezi nula a jedna.

\begin{lstlisting}[language=Python, caption=Kód generující iterpolovanou výškovou mapu, label=kod_interpolovane_pole]
def interpolovane_pole(big_rows, big_cols, small_rows, small_cols, min_value=0, max_value=1):
    # Vytvoření menší mřížky
    small_grid = nahodne_pole(small_rows, small_cols, min_value, max_value)
    # Indexy pro malou a velkou mřížku
    small_x = np.linspace(0, big_cols - 1, small_cols)
    small_y = np.linspace(0, big_rows - 1, small_rows)
    big_x = np.arange(big_cols)
    big_y = np.arange(big_rows)
    # Vytvoření seznamu souřadnic
    small_points = np.array([(x, y) for y in small_y for x in small_x])
    small_values = small_grid.flatten()
    big_points = np.array([(x, y) for y in big_y for x in big_x])
    # Doplnění hodnot seznamu interpolací
    big_list = griddata(small_points, small_values, big_points, method='cubic')
    # Převedení na mřížku
    big_grid = big_list.reshape(big_rows, big_cols)
    return big_grid

\end{lstlisting}

Interpolace vytvoří plynulejší terén bez ostrých přechodů mezi výškami, ale výsledky mají stále tendenci být hranaté a ne zcela přirozeně vypadájící.

\subsection{Gradientní šum}

Gradientní šum je komplikovanější metoda, která vytváří přirozenější struktury terénu než předchozí dvě metody. Konkrétně se inspiruji metodou \textbf{Perlinův šum}, který generuje plynulé změny hodnot v prostoru, čímž vytváří realistické krajiny s kopečky, údolími a horami.

Perlinův šum funguje na principu interpolace gradientních vektorů. Výsledkem je spojitá mapa hodnot, kde se výška plynule mění bez ostrých přechodů, ale vznikají přirozeně vypadající útvary. 

Tuto metodu jsem prvně implementoval pomocí knihovny \textit{noise} jako vzorový výsledek a následně jsem si také napsal vlastní implementaci.

\subsubsection{Vlastní implementace}

Mnou napsaná funkce generuje Perlinův šum v několika krocích:

\begin{enumerate}
    \item Vytvoření mřížky gradientových vektorů -- Každému bodu v hrubé mřížce (menší ze dvou mřížek, vytvořené podle zvolené hodnoty \textit{scale}) je přiřazen náhodný gradientový vektor. Tento vektor určuje, jak se bude hodnota šumu měnit v okolí daného bodu.
    \item Výpočet skalárních součinů -- Pro každý bod na jemné mřížce se vypočítá skalární součin mezi gradientním vektorem a vektorem k bodu, jehož hodnotu chceme určit.
    \item Použití funkce fade -- Hodnoty jsou vyhlazeny pomocí speciální funkce fade, která zajišťuje plynulý přechod mezi body mřížky a zabraňuje vzniku ostrých přechodů.
    \item Interpolace mezi sousedními hodnotami -- Hodnoty se interpolují mezi čtyřmi nejbližšími gradientními body, čímž vznikne plynulý přechod mezi oblastmi s různou výškou.
    \item Normalizace -- Nakonec se hodnoty normalizují aby výsledné hodnoty dobře vycházeli mezi hodnoty nula a jedna.
\end{enumerate}

Použitím různých měřítek (\textit{scale}) je možné ovlivnit rozmanitost a velikost útvarů vygenerované krajiny.

Výslednou matici hodnot, lze použít jako výškovou mapu pro tvorbu herního terénu. Výsledná mapa má plynulé přechody mezi různými výškami a vytváří přirozeně vypadající krajinu.

\subsection{Přiřazení typů terénu}

Po vygenerování výškové mapy je potřeba převést hodnoty na jednotlivé typy terénu, k tomu je použita funkce \texttt{cislo\_na\_policko} \ref{kod_cislo_na_policko}. 

\begin{lstlisting}[language=Python, caption=Převádějící výškovou mapu na konkrétní terény, label=kod_cislo_na_policko]
def cislo_na_policko(grid):
    mapa = np.empty_like(grid, dtype='str')
    for i in range(grid.shape[0]):  # Počet řádků
        for j in range(grid.shape[1]):  # Počet sloupců
            if grid[i][j] < 0.25:
                mapa[i][j] = "V"  # Voda
            elif grid[i][j] < 0.5:
                mapa[i][j] = "P"  # Pláně
            elif grid[i][j] < 0.75:
                mapa[i][j] = "L"  # Les
            else:
                mapa[i][j] = "H"  # Hory
    return mapa

\end{lstlisting}

Aktuální prahové hodnoty jsem zvolil experimentálně, ale mohou být upravovány po testování v simulaci, aby byl terén vyvážený a poskytoval vhodné herní prostředí.

\subsection{Vizualizace mapy}

Matici terénů vycházející z funkce \texttt{cislo\_na\_policko} pak zobrazuji pomocí \textit{matplotlib} pomocí funkce \texttt{zobraz\_mapu} \ref{kod_zobraz_mapu}.

\begin{lstlisting}[language=Python, caption=Kód generující pole pro náhodnou mapu, label=kod_zobraz_mapu]
def zobraz_mapu(mapa):
    """
    Barevně vykreslí terénní mapu.
    """
    # Definice barev pro jednotlivé typy terénu
    barvy = {
        "V": "#1f77b4",  # Modrá - Voda
        "P": "#58d162",  # Světle zelená - Pláně
        "L": "#0c3b10",  # Zelená - Les
        "H": "#4a4a48",  # Šedá - Hory
    }

    # Převedení mapy na numerickou matici s indexy
    text_to_index = {"V": 0, "P": 1, "L": 2, "H": 3}
    index_map = np.vectorize(text_to_index.get)(mapa)

    # Vytvoření barevné mapy
    cmap = ListedColormap(barvy.values())

    # Vykreslení mřížky
    plt.figure(figsize=(8, 8))
    plt.imshow(index_map, cmap=cmap, interpolation='nearest')

    plt.show()

\end{lstlisting}

Výstupem je čtyřbarevný graf složený ze čtvercových dlaždic.

